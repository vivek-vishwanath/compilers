\documentclass[12pt]{article}
\usepackage[utf8]{inputenc}
\usepackage{fancyhdr}
\usepackage[includehead, margin=2cm]{geometry}
\usepackage{listings}
\usepackage{titling}
\usepackage{parskip}
\usepackage{fancyvrb}
\usepackage{tabularx}
\usepackage{float}
\usepackage{hyperref}

\pagestyle{fancy}

\newenvironment{centeredcode}
{
\ttfamily
\begin{center}
\begin{tabular}{l}
}
{
\end{tabular}
\end{center}
}

\title{CS 4240: Compilers and Interpreters, Project 1, Fall 2025\\
\Large
Assigned: August 20, 2025, 10\% of course grade\\
(100 points total)\\
Due by 11:59pm on September 17, 2025
}
\preauthor{}
\postauthor{}
\author{}
\date{}

\begin{document}
\lhead{CS 4240, Fall 2025}
\rhead{Project 1}
\maketitle
\thispagestyle{fancy}

\section{Project Description}

In this semester, through 3 projects, we will build a compiler for a
small pedagogic programming language named Tiger,
targeting the MIPS32 architecture,
through an intermediate representation called Tiger-IR.
The three projects will be:
\begin{itemize}
\item Project 1 (middle end): Static analysis and code optimization of Tiger-IR
\item Project 2 (back end): MIPS32 code generation from Tiger-IR
\item Project 3
\begin{itemize}
  \item Option 1 (front end): Generate a parser for Tiger source code
\item Option 2 (program analysis): Control-flow analysis based on LLVM
  \end{itemize}
\end{itemize}
Each of these projects will have a file input and output interface, to
enable them to be implemented independently.  You may choose any
of the following implementation languages: Java, Python, C, or C++. You must provide complete
self-contained instructions in your submission on how to build and run
your project (including version numbers of programming system and
support libraries), as well as the \texttt{build.sh} and \texttt{run.sh} scripts to do so.  We've provided helper code in Java which may make
it more convenient for teams that use Java as their implementation
language.  Teams that use other languages may need to port some/all of
our Java helper code to their implementation language.

In this first project, you will build a simple optimizer (middle end) for Tiger-IR. 
The reference manual for Tiger-IR can be found in the course project repository
mentioned in Section~\ref{sec:code}.
Your optimizer must be in one of the following languages: Java, Python, C, or C++. The optimizer should read in a text-formatted Tiger-IR program, perform optimizations on it,
and write the optimized IR code to a file.

The goal of your optimizer is to improve the performance of a given program.
In this project, we measure the performance of an IR program by collecting the \emph{dynamic instruction count},
which is the number of instructions executed by an IR interpreter during an execution.
Please note that this is a different goal from minimizing the number of instructions in a program,
which is called the \emph{static instruction count},
as sometimes it is possible to insert more static instructions to
obtain a reduced dynamic instruction count.
As indicated below, we identify a reference optimization (dead code
elimination using reaching definitions) that suffices to obtain full
credit.  However, there is no restriction on which optimizations your optimizer should
contain, as long as they are correct, and meet the dynamic instruction count
performance goals outlined below in Section~\ref{sec:test}.

The deliverables for the project are a design document and code
submission, which are described below in Sections~\ref{sec:opt} and \ref{sec:test}.

\subsection{Design of your Optimizer (30 points)}\label{sec:opt}

With your submission, please include a design document called
`design.pdf' in the zip file mentioned in Section \ref{sec:test}.
%
This document should briefly describe the following:
%
\begin{enumerate}
\item High-level architecture of your optimizer, including the
  analysis and optimization
  algorithm(s) implemented, and why you chose that approach.
\item Low-level design decisions you made in selection of
  implementation language, and their rationale.
\item Software engineering challenges and issues that arose and how
  you resolved them.
\item Any known outstanding bugs or deficiencies that you were unable
  to   resolve before the project submission.
       \item Build and usage instructions for your optimizer.
        \item A summary of the test results for the public test cases
          (see Section~\ref{sec:test}).
 \end{enumerate}
   
\subsection{Evaluation (70 points)}\label{sec:test}

The grading on the performance of your optimizer's output is based on test cases.
A test case is defined as a \textless test-IR-program, test-input\textgreater~pair.
First we will execute your \texttt{build.sh} script to build your optimizer executable.
Then we will run your optimizer on a set of test programs using your \texttt{run.sh} script.  20 points will
be allocated for {\em public} test cases, which we will provide you, and 50
points will be allocated for {\em hidden} test cases, which we will
use when grading your project.  For each test case, we will perform
two steps to evaluate your optimizer:
\begin{enumerate}
\item Execute your  \texttt{run.sh} script to run your optimizer on a test-IR-program to obtain an optimized IR program.
\item Run the optimized IR programs on an IR interpreter (which we will
  provide you for testing purposes) with test-input, and report the
  dynamic instruction count obtained during the execution.  (Label
  instructions will not be included in the dynamic instruction count.)
\end{enumerate}

In any of the following cases, you will get 0\% of the score for a test case:
\begin{itemize}
\item Your optimizer crashes, or produces invalid code.
\item The optimized program does not produce the desired output when
  being interpreted.
\item The optimized program produces the desired output, but its
  dynamic instruction count is $\geq$ that of the original IR program.
\end{itemize}

To get 50\% of the score for a test case, the IR code generated by your optimizer will need to have the same or better performance compared with an optimizer (optimizer A) that does the following optimization:
\begin{itemize}
\item Simple dead (useless) code elimination without branch removal
\begin{itemize}
\item Based on the algorithm in Lecture 1 slides (without reaching definitions)
\end{itemize}
\end{itemize}

To get 100\% of the score for a test case, the IR code generated by
your optimizer will need to have the same or better performance
compared with an optimizer (optimizer B) that does the following
optimization:
\begin{itemize}
\item Dead (useless) code elimination without branch removal
\begin{itemize}
\item Based on the algorithm in  Lecture 3 slides (with reaching definitions)
\end{itemize}
\end{itemize}


The optimizations mentioned above are what we expect most project
teams to implement.  However, you are welcome to implement other/additional
optimizations that you choose.  
Your goal is to make your optimizer generate equivalent or better code than
optimizer A or B does,
in terms of performance (dynamic instruction count).  Assuming
successful completion.  The actual score
that you receive for a test case will be an interpolation between the
50\% and 100\% reference points mentioned above.

\section{Provided Code}\label{sec:code}

We have provided some Java helper code and the IR interpreter in:\\
\url{https://github.gatech.edu/CS-4240-Fall-2025/Project-1}

You should have access if you have activated your GT GitHub account.
If you have not done so, please sign into \url{https://github.gatech.edu}. 
If you can't access the repository, please make a private post on piazza to let the TAs know.

\subsection{Helper Code}

The Java helper code is located under \texttt{/Project-1/materials/src/ir}.
It defines some data structures to represent the IR in memory,
and provides functionality for reading and printing the IR.

If you decide to use the helper code in your project, please be sure
to carefully read all the source files, except \texttt{IRReader.java} and \texttt{IRPrinter.java}.
For these two files, you only need to know how to use the APIs they provide,
which is demonstrated in \texttt{/Project-1/materials/src/Demo.java}, but you are
welcome to read the two files also.
The \texttt{Demo.java} file also includes some important documentation and usage patterns of the helper code.

It is not required to use the helper code, and you are welcome to
modify the helper code as you see fit.

\subsection{Tiger-IR Interpreter}

The source code of the IR interpreter is in \texttt{/Project-1/materials/src/IRInterpreter.java}.
It also uses the helper code.
Please refer to \texttt{/Project-1/materials/README.md} for more information about how to build and use it.
We will grade your project with the same IR interpreter, but you can
feel free to modify it for debugging purposes as needed.

\section{Submission}

The deliverable files are:
\begin{itemize}
    \item The complete source code of your project, as described in Section~\ref{sec:test}.
    \item A \texttt{build.sh} script in the top level directory which builds your project.
    \item A \texttt{run.sh}  script in the top level directory which runs your project executable on an input path to an ir file \texttt{path/to/file.ir}  and outputs an \texttt{out.ir}  file.
    \item The design.pdf file described in Section~\ref{sec:test}.
\end{itemize}

\textbf{More details about submission will be announced on Piazza once the autograder has been set up. Please be on the lookout for that.}

\section{Collaboration}
%
We will award identical grades to each member of a given project team, unless members of the team directly register a formal complaint. Should any team member wish to join a new group for Project-X (X=1,2,3), they are required to communicate their request to both fellow team members and the teaching staff within one week of the corresponding Project-X release date.
%
We assume that the work submitted by each team is their work solely.
% 
Any clarification question about the project handout should be posted
on the course's public Piazza message
board.
%
Any non-obvious discussion or questions about design and
implementation should be either posted on the course's Piazza message
boards privately for the instructors or presented in person during
office hours.
%
If the instructors determine that parts of the discussion are
appropriate for the entire class, then they will forward selections.
%
Under no condition is it acceptable to use code written by another
team, or obtained from any other source.
%
As part of the standard grading process, each submitted solution will
automatically be checked for similarity with other submitted solutions and with other known
implementations.

\end{document}
